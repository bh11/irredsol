%%%%%%%%%%%%%%%%%%%%%%%%%%%%%%%%%%%%%%%%%%%%%%%%%%%%%%%%%%%%%%%%%%%%%%%%%%%%%
%%
%%  overview.tex         IRREDSOL documentation           Burkhard H\"ofling
%%
%%  @(#)$Id$
%%
%%  Copyright (C) 2003 by Burkhard H\"ofling, 
%%  Institut f\"ur Geometrie, Algebra und Diskrete Mathematik
%%  Technische Universit\"at Braunschweig, Germany
%%
%%%%%%%%%%%%%%%%%%%%%%%%%%%%%%%%%%%%%%%%%%%%%%%%%%%%%%%%%%%%%%%%%%%%%%%%%%%%%
\Chapter{Overview}


Let $n$ be a positive integer and $q$ a prime power satisfying  $q^n \leq
2^{16}-1$. The package {\IRREDSOL} constitutes a library of irreducible
solvable subgroups of $GL(n, q)$, such that each irreducible solvable
subgroup of
$GL(n, q)$ is conjugate to precisely one group in the library. 

This data base is
intended to replace and extend  the data base of irreducible matrix groups
over prime fields created by Mark Short \cite{Sho} which is also part of
{\GAP}. There are functions available to translate from Mark Short's
numbering of groups to the numbering used in {\IRREDSOL} and back; see
"Compatibility with other data libraries".

It is possible to construct groups in this list one at a time (see "low
level access functions"), by supplying four integer parameters identifying
the group in question. In addition, there are functions which facilitate 
searching the library for groups with given properties (see "Finding
matrix groups with given properties").

Given an irreducible solvable matrix group <G>, it is possible
to find the group in the library to which <G> is conjugate, see
"identification of irreducible groups" and "identification of absolutely
irreducible groups".

Finally, the {\IRREDSOL} package provides additional functionality
for matrix groups, such as the computation of imprimitivity systems.

The methods used to construct the data base are slight modifications of
those described by Bettina Eick and the author in \cite{EH}.

%%%%%%%%%%%%%%%%%%%%%%%%%%%%%%%%%%%%%%%%%%%%%%%%%%%%%%%%%%%%%%%%%%%%%%%%%%%%%
%%
%E  
%%
