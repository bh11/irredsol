%!TEX root = manual.in.tex
%%%%%%%%%%%%%%%%%%%%%%%%%%%%%%%%%%%%%%%%%%%%%%%%%%%%%%%%%%%%%%%%%%%%%%%%%%%%%
%%
%%  overview.tex           IRREDSOL documentation            Burkhard Höfling
%%
%%  Copyright © 2003–2016 Burkhard Höfling 
%%
%%%%%%%%%%%%%%%%%%%%%%%%%%%%%%%%%%%%%%%%%%%%%%%%%%%%%%%%%%%%%%%%%%%%%%%%%%%%%
\Chapter{Overview}

\index{IRREDSOL}
\index{structure of IRREDSOL}

The package {\IRREDSOL} provides a library of irreducible
soluble subgroups of matrix groups over finite fields and a corresponding library of primitive soluble groups.

Currently, {\IRREDSOL} contains all subgroups, up to conjugacy, of $GL(n, q)$, 
where $n$ is a positive integer and $q$
is a prime power satisfying  $q^n \leq 2^{24} - 1 = 16\,777\,215$. The underlying data base consists of 
$ 921\,371$ absolutely irreducible groups of degree $n > 1$ amounting to $1\,089\,136$ irreducible groups of degree~$n>1$. See Section~"Design of the group library" for details.

The groups in the {\IRREDSOL} 
library can be accessed one at a time (see Section~"Low
level access functions"). In addition, there are functions which allow to 
search the library for groups with given properties (see Section "Finding
matrix groups with given properties"). Moreover, given an irreducible soluble matrix group
<G>, it is possible to identify the group in the library to which <G> is conjugate,
including a conjugating matrix, if desired. See Section~"Identification of irreducible
groups".

Apart from this, the {\IRREDSOL} package provides additional functionality
for matrix groups, such as the computation of imprimitivity systems;
see Chapter~"Additional functionality for matrix groups".

It is well-known that there is a bijection between the  irreducible soluble subgroups of
$GL(n, p)$, where
$p$ is a prime, and the conjugacy classes, or equivalently the isomorphism types, of
primitive soluble subgroups of ${\rm Sym}(p^n)$. The {\IRREDSOL} package contains
functions to translate between irreducible soluble matrix groups and primitive
groups, to search for primitive soluble groups with given  properties, and functions to
recognise them, up to isomorphism (or, equivalently, up to conjugacy in ${\rm Sym}(p^n)$).  See Sections "Converting between irreducible soluble
matrix groups and primitive soluble groups", "Finding primitive soluble permutation
groups with given properties", and "Recognising primitive soluble groups", respectively.

Note that {\GAP} contains another library consisting of all $372$ irreducible soluble
subgroups of $GL(n, p)$, where $n > 1$, $p$ is a prime, and $p^n \< 2^8$. This library 
was originally
created by Mark Short~\cite{Sho}, and two omissions in $GL(2,13)$ were added later; 
see Section "ref:Irreducible Solvable Matrix Groups" in the {\GAP} reference manual.
All of these groups are,  of course, also part of the {\IRREDSOL} data base, and the
{\IRREDSOL} package provides functions to identify the groups in the
{\GAP} library in {\IRREDSOL} and viceversa. See
Section~"Compatibility with other data libraries".

The groups in the {\IRREDSOL} data base were constructed using the Aschbacher 
classification~\cite{Asc} of maximal subgroups of linear groups. Further details can be found 
in~\cite{EH}, where the 
construction of all irreducible soluble subgroups of $GL(n, q)$ with $q^n \< 3^8$
is described.

For a historical account of the classification of irreducible matrix groups and
primitive permutation groups, the reader is referred to \cite{Sho} and, 
for more recent developments, to~\cite{EH}.


%%%%%%%%%%%%%%%%%%%%%%%%%%%%%%%%%%%%%%%%%%%%%%%%%%%%%%%%%%%%%%%%%%%%%%%%%%%%%
%%
%E  
%%
