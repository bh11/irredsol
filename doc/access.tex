%%%%%%%%%%%%%%%%%%%%%%%%%%%%%%%%%%%%%%%%%%%%%%%%%%%%%%%%%%%%%%%%%%%%%%%%%%%%%
%%
%%  access.tex            IRREDSOL documentation           Burkhard Hoefling
%%
%%  @(#)$Id$
%%
%%  Copyright (C) 2003-2005 by Burkhard Hoefling, 
%%  Institut fuer Geometrie, Algebra und Diskrete Mathematik
%%  Technische Universitaet Braunschweig, Germany
%%


%%%%%%%%%%%%%%%%%%%%%%%%%%%%%%%%%%%%%%%%%%%%%%%%%%%%%%%%%%%%%%%%%%%%%%%%%%%%%
\Chapter{Accessing the data library}

This chapter describes the design of the {\IRREDSOL} group library (see 
Section "Design of the group library") and the
 various ways of accessing groups in the data library.
It is possible to access individual groups in the group library (see
Section~"Low level access functions"), or to search for groups with certain
properties (see Section~"Finding matrix groups with given properties"). 
Finally, there are functions for loading and unloading group data 
manually (see Section~"Loading and unloading group data manually").


%%%%%%%%%%%%%%%%%%%%%%%%%%%%%%%%%%%%%%%%%%%%%%%%%%%%%%%%%%%%%%%%%%%%%%%%%%%%%
\Section{Design of the group library}

To avoid redundancy, the package {\IRREDSOL} does not actually store 
lists of irreducible subgroups of $GL(<n>, <q>)$ but
only has lists ${\cal
A}_{n,q}$ of subgroups of $GL(<n>, <q>)$ such that

\beginlist
\item{--} each group in ${\cal A}_{n,q}$ is absolutely irreducible and
solvable

\item{--} ${\cal A}_{n,q}$ contains a conjugate of each absolutely
irreducible solvable subgroup of $GL(n, q)$

\item{--} no two groups in ${\cal A}_{n,q}$ are conjugate

\item{--} each group in ${\cal A}_{n,q}$ has trace field $GF(q)$.

\endlist Here, ``conjugate'' means ``conjugate in $GL(n,q)$''. We will
briefly say that ${\cal A}_{n,q}$ contains, up to conjugacy, all absolutely
irreducible solvable subgroups of $GL(n,q)$ with trace field $GF(q)$. 
Here, the {\it trace field} of a
subgroup $G$ of
$GL(n, q)$ is the field generated by the traces of elements of~$G$. By a 
theorem of Brauer, an irreducible subgroup of $GL(n, q)$ with trace field $GF(q_0)$ 
has a conjugate lying in $GL(n, q_0)$. See also `TraceField' ("TraceField")
and `ConjugatingMatTraceField' ("ConjugatingMatTraceField").

These lists are sufficient to reconstruct lists of irreducible solvable
subgroups of $GL(n, q)$, since  any such subgroup $G$ can be obtained from an
absolutely irreducible subgroup of $GL(n/d, q^d)$, where $d$ divides $n$,
by regarding the underlying $\F_{q^d}$-vector space as an $\F_q$-vector
space. Here, $\F_{q^d}$ is the splitting field for $G$.
Note that two subgroups of $GL(n, q)$
constructed in that way are conjugate if, and only if, the images of the corresponding subgroups
of $GL(n/d, q^d)$ under suitable Galois automorphism of $\F_{q^d}/\F_q$ are conjugate in $GL(n/d,
q^d)$. This information also forms part of the {\IRREDSOL} library.

Note that by the Deuring-Noether theorem, two subgroups of $GL(n, q)$ are
conjugate in $GL(n, q)$ if, and only if, they are conjugate in $GL(n, q^d)$
for some $d > 1$. This ensures that lists of irreducible subgroups obtained
from the ${\cal A}_{n,q}$ do not contain conjugate subgroups. 


%%%%%%%%%%%%%%%%%%%%%%%%%%%%%%%%%%%%%%%%%%%%%%%%%%%%%%%%%%%%%%%%%%%%%%%%%%%%%
\Section{Low level access functions}

The access functions described in this section allow one
to check for the availability of data and the construction of irreducible
irreducible  groups in the {\IRREDSOL} group library.

\>`IsAvailableIrreducibleSolvableGroupData(<n>, <q>)'%
{IsAvailableIrreducibleSolvableGroupData}%
@{`IsAvailableIrreducibleSolvable\\GroupData'} F

This function tests whether the irreducible solvable subgroups of $GL(n,q)$ with trace
field $\F_q$ are part of the {\IRREDSOL} library.


\>IndicesIrreducibleSolvableMatrixGroups(<n>, <q>, <d>) F

Let <n>, <d> be positive integers and <q> a prime power. This
function returns a set of integers parametrising the groups in the {\IRREDSOL} library
which are subgroups of $GL(n,q)$ with trace field $\F_q$
and have splitting field $\F_{q^d}$. This set is empty unless <d> divides <n>. An error is 
raised if the relevant data is not available (see "IsAvailableIrreducibleSolvableGroupData" 
for information how to check this first).


\>IrreducibleSolvableMatrixGroup(<n>, <q>, <d>, <k>) F

Let <n> be a  positive integer and <q> a prime power. This
function returns the <k>-th irreducible solvable subgroup of $GL(n,q)$ with trace field $\F_q$ 
and has splitting field~$\F_{q^d}$.
An error is raised if the relevant
data is not available (see "IsAvailableIrreducibleSolvableGroupData" for information 
how to check this first), or if <k> is not in  
`IndicesIrreducibleSolvableMatrixGroups'(<n>, <q>, <d>) 
(see "IndicesIrreducibleSolvableMatrixGroups").
For the groups returned, the attributes and properties described in
Chapter "Additional functionality for matrix groups" are set to their appropriate values.


\>`IsAvailableAbsolutelyIrreducibleSolvableGroupData(<n>, <q>)'%
{IsAvailableAbsolutelyIrreducibleSolvableGroupData}%
@{`IsAvailableAbsolutelyIrreducibleSolvable\\GroupData'} F

This function tests whether the
absolutely irreducible solvable subgroups of $GL(n,q)$ with trace field $\F_q$ are in the {\IRREDSOL} library.


\>`IndicesAbsolutelyIrreducibleSolvableMatrixGroups(<n>, <q>)'%
{IndicesAbsolutelyIrreducibleSolvableMatrixGroups}%
@{`IndicesAbsolutelyIrreducibleSolvable\\MatrixGroups'} F

Let <n> be a  positive integer and <q> a prime power. This
function returns a set of integers parametrising the absolutely irreducible groups in the
{\IRREDSOL} library which are subgroups of $GL(n,q)$ with trace field~$\F_q$.

\beginexample
gap> IndicesIrreducibleSolvableMatrixGroups (6, 2, 2);
[ 1, 2, 3, 4, 5, 6, 7, 8, 10, 11, 12 ]
\endexample

\>AbsolutelyIrreducibleSolvableMatrixGroup(<n>, <q>, <k>) F

Let <n> be a  positive integer and <q> a prime power. This
function returns the <k>-th absolutely irreducible solvable subgroup of $GL(n,q)$ with trace 
field~$\F_q$. An error is raised if the relevant
data is not available (see "IsAvailableAbsolutelyIrreducibleSolvableGroupData" for information 
how to check this first), or if <k> does not lie in
`IndicesAbsolutelyIrreducibleSolvableMatrixGroups'(<n>, <q>)
(see "IndicesAbsolutelyIrreducibleSolvableMatrixGroups"). For the groups returned, 
the attributes and properties described in Chapter~"Additional functionality for matrix groups" 
are set to their appropriate values.


\>`IndicesMaximalAbsolutelyIrreducibleSolvableMatrixGroups(<n>, <q>)'%
{IndicesMaximalAbsolutelyIrreducibleSolvableMatrixGroups}%
@{`IndicesMaximalAbsolutelyIrreducible\\Solvable\\MatrixGroups'} F

Let <n> be a  positive integer and <q> a prime power. This
function returns a set of integers parametrising those absolutely irreducible groups in the
{\IRREDSOL} library that are subgroups of $GL(n,q)$ with trace field~$\F_q $ and that are maximal with
respect to being solvable. An error is raised if the relevant
data is not available (see "IsAvailableAbsolutelyIrreducibleSolvableGroupData" for information 
how to check this first).

\beginexample
gap> inds := IndicesMaximalAbsolutelyIrreducibleSolvableMatrixGroups (2,3);
[ 2 ] 
gap> max := AbsolutelyIrreducibleSolvableMatrixGroup (2,3,2);
Group([ [ [ Z(3), 0*Z(3) ], [ 0*Z(3), Z(3)^0 ] ], [ [ Z(3)^0, Z(3) ], [ 0*Z(3), Z(3)^0 ] ], 
  [ [ Z(3), Z(3)^0 ], [ Z(3)^0, Z(3)^0 ] ], [ [ 0*Z(3), Z(3)^0 ], [ Z(3), 0*Z(3) ] ], 
  [ [ Z(3), 0*Z(3) ], [ 0*Z(3), Z(3) ] ] ])
gap> max = GL(2,3); # it is the whole GL
true
\endexample


%%%%%%%%%%%%%%%%%%%%%%%%%%%%%%%%%%%%%%%%%%%%%%%%%%%%%%%%%%%%%%%%%%%%%%%%%%%%%
\Section{Finding matrix groups with given properties}

This section describes three functions
(`AllIrreducibleSolvableMatrixGroups',
`OneIrreducibleSolvableMatrixGroup',
`IteratorIrreducibleSolvableMatrixGroups') which allow you to find matrix
groups with special properties. Using these functions can be more efficient
than to construct each group in the library using the functions in Section
"Low level access functions" because they can access additional information 
about a group in the {\IRREDSOL} library before constructing the group itself. 
See the discussion following the description of 
`AllIrreducibleSolvableMatrixGroups' for details. 

\>AllIrreducibleSolvableMatrixGroups(<func_1>, <arg_1>, <func_2>, <arg_2>, \dots) F

This function returns a list of all irreducible solvable matrix
groups <G> in the {\IRREDSOL} library for which the return value of $<func_i>(G)$ lies in
<arg_i>.  The arguments <func_1>, <func_2>, \dots,
must be {\GAP} functions which take a matrix group as their only argument and
return a value, and <arg_1>, <arg_2>,
\dots,  must be lists. If <arg_i> is not a list, <arg_i> is replaced by the list
`[<arg_i>]'. One of the functions must be `DegreeOfMatrixGroup' (or one of its
equivalents, see below), and one function must be  `FieldOfMatrixGroup' (or `Field' or 
`TraceField'). Note that all groups in the data library have the property that 
$`TraceField(<G>)' = `FieldOfMatrixGroup'(<G>)$; see Section~"Design of the group library" 
for details. 
For the groups returned, the attributes and properties described in 
Chapter~"Additional functionality for matrix groups" are set to their appropriate values.


Note that there is also a function `IteratorIrreducibleSolvableMatrixGroups' (see
"IteratorIrreducibleSolvableMatrixGroups") which allows to run through the list produced by
`AllIrreducibleSolvableMatrixGroups' without having to store all of the groups
simultaneously.

The following functions <func_i> are handled particularly efficiently, because the
return values of these functions can be read off the
{\IRREDSOL} library without actually constructing the relevant matrix group. For the
definitions of these functions, see Chapter~"Additional functionality for matrix
groups" and "ref:Size" in the {\GAP} reference manual.

\beginlist

\item{--} `DegreeOfMatrixGroup' (or `Degree', `Dimension', `DimensionOfMatrixGroup'), 
\item{--} `CharacteristicOfField' (or `Characteristic')
\item{--} `FieldOfMatrixGroup' (or `Field' or `TraceField')
\item{--} `Order' (or `Size')
\item{--} `IsMaximalAbsolutelyIrreducibleSolvableMatrixGroup' 
\item{--} `IsAbsolutelyIrreducibleMatrixGroup' (or `IsAbsolutelyIrreducible')
\item{--} `MinimalBlockDimensionOfMatrixGroup' (or `MinimalBlockDimension')
\item{--} `IsPrimitiveMatrixGroup' (or `IsPrimitive', `IsLinearlyPrimitive')

\endlist
Note that computations with matrix groups over large vector spaces tend to be slow in {\GAP}. 
If you wish to investigate many groups, you may speed up computations by computing with
an isomorphic copy of <G>. Such an isomorphism can be found in the attribute 
`RepresentationIsomorphism' (see "RepresentationIsomorphism").

\beginexample
# get just those groups with trace field GF(9)
gap> l := AllIrreducibleSolvableMatrixGroups (Degree, 1, Field, GF(9));;
gap> List (l, Order);
[ 4, 8 ]
# get all irreducible subgroups
gap> l := AllIrreducibleSolvableMatrixGroups (Degree, 1, Field, Subfields (GF(9)));;
gap> List (l, Order);
[ 1, 2, 4, 8 ]
# get only maximal absolutely irreducible ones
gap> l := AllIrreducibleSolvableMatrixGroups (Degree, 4, Field, GF(3),
>             IsMaximalAbsolutelyIrreducibleSolvableMatrixGroup, true);;
gap> SortedList (List (l, Order));
[ 320, 640, 2304, 4608 ]
gap> l := AllIrreducibleSolvableMatrixGroups (Degree, 4, Field, GF(3),
> IsAbsolutelyIrreducibleMatrixGroup, true);;
gap> Collected (List (l, Order));
[ [ 20, 1 ], [ 32, 7 ], [ 40, 2 ], [ 64, 10 ], [ 80, 2 ], [ 96, 6 ], 
  [ 128, 9 ], [ 160, 3 ], [ 192, 9 ], [ 256, 6 ], [ 288, 1 ], [ 320, 2 ], 
  [ 384, 4 ], [ 512, 1 ], [ 576, 3 ], [ 640, 1 ], [ 768, 1 ], [ 1152, 4 ], 
  [ 2304, 3 ], [ 4608, 1 ] ]
\endexample

\>OneIrreducibleSolvableMatrixGroup(<func_1>, <arg_1>, <func_2>, <arg_2>, \dots) F

This function returns a matrix group <G> from the {\IRREDSOL} library such that
$<func_i>(G)$ lies in <arg_i>, or `fail' if no such group exists. The arguments <func_1>,
<func_2>, \dots, must be {\GAP} functions taking one argument and returning a value, and
<arg_1>, <arg_2>, \dots,  must be lists. If <arg_i> is not a list, <arg_i> is replaced by
the list `[<arg_i>]'. One of the functions must be `DegreeOfMatrixGroup' (or one of its
equivalents, see below), and one function must be  `FieldOfMatrixGroup' (or `Field' or 
`TraceField'). Note that all groups in the data library have the property that 
$`TraceField(<G>)' = `FieldOfMatrixGroup'(<G>)$; see Section~"Design of the group library" 
for details. 
For the group returned, the attributes and properties described in 
Chapter~"Additional functionality for matrix groups" are set to their appropriate values.

For a list of functions which are handled particularly efficiently, see
"AllIrreducibleSolvableMatrixGroups".

\>IteratorIrreducibleSolvableMatrixGroups(<func_1>, <arg_1>, <func_2>, <arg_2>, \dots) F

This function returns an iterator which runs through the list of all matrix groups <G>
in the  {\IRREDSOL} library such that
$<func_i>(G)$ lies in <arg_i>. The arguments <func_1>, <func_2>, \dots,
must be {\GAP} functions taking one argument and returning a value, and <arg_1>, <arg_2>, \dots, 
must be lists. If <arg_i> is not a list, <arg_i> is replaced by the list `[<arg_i>]'.
One of the functions must be `DegreeOfMatrixGroup' (or `Degree', `Dimension', 
or`DimensionOfMatrixGroup'), and one
function must be  `FieldOfMatrixGroup' (or `Field'). 

For a list of functions which are handled particularly efficiently, see
"AllIrreducibleSolvableMatrixGroups".

Using 

`IteratorIrreducibleSolvableMatrixGroups'(<func_1>, <arg_1>, <func_2>, <arg_2>, \dots)) 

is functionally equivalent to 

`Iterator'(`AllIrreducibleSolvableMatrixGroups'(<func_1>, <arg_1>, <func_2>, <arg_2>, \dots))

(see "ref:Iterators" in the {\GAP} reference manual for details) but does not compute all 
relevant matrix groups at the same time. This may save some memory. 
For the groups returned, the attributes and properties described in
Chapter~"Additional functionality for matrix groups" are set to their appropriate values.


%%%%%%%%%%%%%%%%%%%%%%%%%%%%%%%%%%%%%%%%%%%%%%%%%%%%%%%%%%%%%%%%%%%%%%%%%%%%%
\Section{Loading and unloading group data manually}
The data required by the {\IRREDSOL} library is loaded into {\GAP}'s workspace automatically whenever required, but is never unloaded automatically. The functions described in this
and the following section describe how to load and unload this data manually. 
They are only relevant if timing or conservation of memory is an issue.
\index{workspace!running out of}

\>LoadAbsolutelyIrreducibleSolvableGroupData(<n>, <q>) F

This function loads the data for $GL(<n>, <q>)$ into memory and does some pre-processing. If the data
is already loaded, the function does nothing. This function is called automatically when you access the
{\IRREDSOL} library, so most users  will not need this function.

\>`LoadedAbsolutelyIrreducibleSolvableGroupData()'%
{LoadedAbsolutelyIrreducibleSolvableGroupData}%
@{`LoadedAbsolutelyIrreducibleSolvable\\GroupData'} F

This function returns a list. Each entry consists of an integer <n> and a set <l>. The set
<l> contains all prime powers <q> such that the group data for $GL(<n>, <q>)$ is currently in memory.

\>`UnloadAbsolutelydIrreducibleSolvableGroupData([<n> [, <q>]])'%
{UnloadAbsolutelydIrreducibleSolvableGroupData}%
@{`UnloadAbsolutelydIrreducibleSolvable\\GroupData'} F

This function can be used to delete data for irreducible groups from the {\GAP} workspace. If no argument
is given, all data will be deleted. If only <n> is given, all data for degree <n> (and any <q>) will
be deleted. If <n> and <q> are given, only the data for $GL(n, q)$ will be deleted from the {\GAP}
workspace. Use this function if you run out of {\GAP} workspace. The
data is automatically re-loaded when required.


%%%%%%%%%%%%%%%%%%%%%%%%%%%%%%%%%%%%%%%%%%%%%%%%%%%%%%%%%%%%%%%%%%%%%%%%%%%%%
%%
%E
%%
