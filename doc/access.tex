%%%%%%%%%%%%%%%%%%%%%%%%%%%%%%%%%%%%%%%%%%%%%%%%%%%%%%%%%%%%%%%%%%%%%%%%%%%%%
%%
%%  access.tex            IRREDSOL documentation           Burkhard H\"ofling
%%
%%  @(#)$Id$
%%
%%  Copyright (C) 2003 by Burkhard H\"ofling, 
%%  Institut f\"ur Geometrie, Algebra und Diskrete Mathematik
%%  Technische Universit\"at Braunschweig, Germany
%%


%%%%%%%%%%%%%%%%%%%%%%%%%%%%%%%%%%%%%%%%%%%%%%%%%%%%%%%%%%%%%%%%%%%%%%%%%%%%%
\Chapter{Accessing the data library}

This chapter describes various ways of accessing groups in the data library.
It is possible to access individual groups in the lists provided  (see
Section  "Low level access functions"), or to search for groups with certain
properties (see Section "Finding matrix groups with given properties"). 


%%%%%%%%%%%%%%%%%%%%%%%%%%%%%%%%%%%%%%%%%%%%%%%%%%%%%%%%%%%%%%%%%%%%%%%%%%%%%
\Section{Low level access functions}

The access functions described in this section allow
to check for the availability of data, the number of groups in each list
${\cal A}_{n,q}$, individual groups in such a 
list, and the corresponding irreducible but not absolutely irreducible 
groups  (see Chapter "Overview").

\>`IsAvailableIrreducibleSolvableGroupData(<n>, <q>)'%
{IsAvailableIrreducibleSolvableGroupData}%
@{`IsAvailableIrreducibleSolvable\\GroupData'} F

This function tests whether the irreducible solvable subgroup of $GL(n,q)$ which
cannot be written over a proper subfield of $\F_q$ are part of the {\IRREDSOL} library.


\>IndicesIrreducibleSolvableMatrixGroups(<n>, <q>, <d>) F

Let <n>, <d> be positive integers and <q> a prime power. This
function returns a set of integers parametrising the groups in the {\IRREDSOL} library
which are subgroups of $GL(n,q)$ that cannot be written over a proper subfield of $\F_q$
and have splitting field $\F_{q^d}$. This set is empty unless <d> divides <n>. An error is raised if the relevant
data is not available (see "IsAvailableIrreducibleSolvableGroupData" for information 
how to check this first).


\>IrreducibleSolvableMatrixGroup(<n>, <q>, <d>, <k>) F

Let <n> be a  positive integer and <q> a prime power. This
function returns the <k>-th irreducible solvable subgroup of $GL(n,q)$ which
cannot be written over a proper subfield of $\F_q$. 
An error is raised if the relevant
data is not available (see "IsAvailableIrreducibleSolvableGroupData" for information 
how to check this first), or if <k> is not in  
`IndicesIrreducibleSolvableMatrixGroups'(<n>, <q>, <d>) 
(see "IndicesIrreducibleSolvableMatrixGroups").
For the groups returned, the attrbutes and properies described in
"Additional functionality for matrix groups" are set to their appropriate values.


\>`IsAvailableAbsolutelyIrreducibleSolvableGroupData(<n>, <q>)'%
{IsAvailableAbsolutelyIrreducibleSolvableGroupData}%
@{`IsAvailableAbsolutelyIrreducibleSolvable\\GroupData'} F

This function tests whether the
absolutely irreducible solvable subgroup of $GL(n,q)$ which
cannot be written over a proper subfield of $\F_q$ are in the {\IRREDSOL} library.


\>`IndicesAbsolutelyIrreducibleSolvableMatrixGroups(<n>, <q>)'%
{IndicesAbsolutelyIrreducibleSolvableMatrixGroups}%
@{`IndicesAbsolutelyIrreducibleSolvable\\MatrixGroups'} F

Let <n> be a  positive integer and <q> a prime power. This
function returns a set of integers parametrising the absolutely irreducible groups in the
{\IRREDSOL} library which are subgroups of $GL(n,q)$ that cannot be written over a proper
subfield of
$\F_q$.


\>AbsolutelyIrreducibleSolvableMatrixGroup(<n>, <q>, <k>) F

Let <n> be a  positive integer and <q> a prime power. This
function returns the <k>-th absolutely irreducible solvable subgroup of $GL(n,q)$ which
cannot be written over a proper subfield of $\F_q$. An error is raised if the relevant
data is not available (see "IsAvailableAbsolutelyIrreducibleSolvableGroupData" for information 
how to check this first), or if <k> does not lie in
`IndicesAbsolutelyIrreducibleSolvableMatrixGroups'(<n>, <q>)
(see "IndicesAbsolutelyIrFor the groups returned, the attrbutes and properies described in
"Additional functionality for matrix groups" are set to their appropriate values.
reducibleSolvableMatrixGroups").


\>`IndicesMaximalAbsolutelyIrreducibleSolvableMatrixGroups(<n>, <q>)'%
{IndicesMaximalAbsolutelyIrreducibleSolvableMatrixGroups}%
@{`IndicesMaximalAbsolutelyIrreducible\\Solvable\\MatrixGroups'} F

Let <n> be a  positive integer and <q> a prime power. This
function returns a set of integers parametrising the absolutely irreducible groups in the
{\IRREDSOL} library which are subgroups of $GL(n,q)$ that cannot be written over a proper
subfield of
$\F_q $and which is maximal with
respect to being solvable. An error is raised if the relevant
data is not available (see "IsAvailableAbsolutelyIrreducibleSolvableGroupData" for information 
how to check this first).

\beginexample
gap> Reread("/Macintosh HD/Applications (Mac OS 9)/gap/4.0/pkg/irredsol/lib/loading.gi");
gap> inds := IndicesMaximalAbsolutelyIrreducibleSolvableMatrixGroups (2,3);
[ 2 ] # there is only one maximal solvable subgroup of GL(2,3)
gap> max := AbsolutelyIrreducibleSolvableMatrixGroup (2,3,2); # construct it
Group([ [ [ Z(3), 0*Z(3) ], [ 0*Z(3), Z(3)^0 ] ], [ [ Z(3)^0, Z(3) ], [ 0*Z(3), Z(3)^0 ] ], 
  [ [ Z(3), Z(3)^0 ], [ Z(3)^0, Z(3)^0 ] ], [ [ 0*Z(3), Z(3)^0 ], [ Z(3), 0*Z(3) ] ], 
  [ [ Z(3), 0*Z(3) ], [ 0*Z(3), Z(3) ] ] ])
gap> max = GL(2,3); # it is the whole GL
true
\endexample


%%%%%%%%%%%%%%%%%%%%%%%%%%%%%%%%%%%%%%%%%%%%%%%%%%%%%%%%%%%%%%%%%%%%%%%%%%%%%
\Section{Finding matrix groups with given properties}

This section describes three functions
(`AllIrreducibleSolvableMatrixGroups',
`OneIrreducibleSolvableMatrixGroup',
`IteratorIrreducibleSolvableMatrixGroups') which allow you to find matrix
groups with special properties. These functions are usually more efficient
than to construct each group in the library using the functions in Section
"Low level access functions".

\>AllIrreducibleSolvableMatrixGroups(<func_1>, <arg_1>, <func_2>, <arg_2>, \dots) F

This function returns a list of all irreducible solvable matrix
groups <G> in the {\IRREDSOL} library for which the return value of $<func_i>(G)$ lies in
<arg_i>.  The arguments <func_1>, <func_2>, \dots,
must be {\GAP} functions which take matrix group as their only argument and
return a value, and <arg_1>, <arg_2>,
\dots,  must be lists. If <arg_i> is not a list, <arg_i> is replaced by the list
`[<arg_i>]'. One of the functions must be `DegreeOfMatrixGroup' (or one of its
equivalents, see below), and one function must be  `FieldOfMatrixGroup' (or `Field'). All
groups <G> in the data library have the property that they cannot be written over a
subfield of $`FieldOfMatrixGroup'(<G>)$; see "Design of the group library" for details. 
For the groups returned, the attrbutes and properies described in
"Additional functionality for matrix groups" are set to their appropriate values.


Note that there is also a function `IteratorIrreducibleSolvableMatrixGroups' (see
"IteratorIrreducibleSolvableMatrixGroups") which allows to run through the list produced by
`AllIrreducibleSolvableMatrixGroups' without having to store all of the groups
simultaneously.

The following functions <func_i> are handled particularly efficiently. For the
definitions of most of these functions, see Chapter~"Additional functionality for matrix
groups".

\beginlist

\item{--} `DegreeOfMatrixGroup' (or `Degree', `Dimension', `DimensionOfMatrixGroup'), 
\item{--} `CharacteristicOfField' (or `Characteristic')
\item{--} `FieldOfMatrixGroup' (or `Field')
\item{--} `Order' (or `Size')
\item{--} `IsMaximalAbsolutelyIrreducibleSolvableMatrixGroup' 
\item{--} `IsAbsolutelyIrreducibleMatrixGroup' (or `IsAbsolutelyIrreducible')
\item{--} `MinimalBlockDimensionOfMatrixGroup' (or `MinimalBlockDimension')
\item{--} `IsPrimitiveMatrixGroup' (or `IsPrimitive', `IsLinearlyPrimitive')

\endlist
Except for groups which are irreducible but not absolutely irreducible and functions
`MinimalBlockDimensionOfMatrixGroup' and `IsPrimitiveMatrixGroup' (or their
equivalents), the return values of the above functions can be read off the
{\IRREDSOL} library without actually constructing the relevant matrix group.
But even in this last case, some groups can be ruled out before constructing
them.

\beginexample
# get just those groups which cannot be written over GF(3)
gap> l := AllIrreducibleSolvableMatrixGroups (Degree, 1, Field, GF(9));;
gap> List (l, Order);
[ 4, 8 ]
# get all irreducible subgroups
gap> l := AllIrreducibleSolvableMatrixGroups (Degree, 1, Field, Subfields (GF(9)));;
gap> List (l, Order);
[ 1, 2, 4, 8 ]
# get only maximal absolutely irreducible ones
gap> l := AllIrreducibleSolvableMatrixGroups (Degree, 4, Field, GF(3),
>             IsMaximalAbsolutelyIrreducibleSolvableMatrixGroup, true);
gap> SortedList (List (l, Order));
[ 320, 640, 2304, 4608 ]
gap> l := AllIrreducibleSolvableMatrixGroups (Degree, 4, Field, GF(3),
> IsAbsolutelyIrreducibleMatrixGroup, true);;
gap> Collected (List (l, Order));
[ [ 20, 1 ], [ 32, 7 ], [ 40, 2 ], [ 64, 10 ], [ 80, 2 ], [ 96, 6 ], 
  [ 128, 9 ], [ 160, 3 ], [ 192, 9 ], [ 256, 6 ], [ 288, 1 ], [ 320, 2 ], 
  [ 384, 4 ], [ 512, 1 ], [ 576, 3 ], [ 640, 1 ], [ 768, 1 ], [ 1152, 4 ], 
  [ 2304, 3 ], [ 4608, 1 ] ]
\endexample

\>OneIrreducibleSolvableMatrixGroup(<func_1>, <arg_1>, <func_2>, <arg_2>, \dots) F

This function returns a matrix group <G> from the {\IRREDSOL} library such that
$<func_i>(G)$ lies in <arg_i>, or `fail' if no such group exists. The arguments <func_1>,
<func_2>, \dots, must be {\GAP} functions taking one argument and returning a value, and
<arg_1>, <arg_2>, \dots,  must be lists. If <arg_i> is not a list, <arg_i> is replaced by
the list `[<arg_i>]'. One of the functions must be `DegreeOfMatrixGroup' (or one of its
equivalents, see below), and one function must be  `FieldOfMatrixGroup' (or `Field'). All
groups <G> in the data library have the property that they cannot be written over a
subfield of $`FieldOfMatrixGroup'(<G>)$; see see "Design of the group library" for details. 
For the group returned, the attrbutes and properies described in
"Additional functionality for matrix groups" are set to their appropriate values.


For a list of functions which are handled particularly efficiently, see
`AllIrreducibleSolvableMatrixGroups' ("AllIrreducibleSolvableMatrixGroups").

\>IteratorIrreducibleSolvableMatrixGroups(<func_1>, <arg_1>, <func_2>, <arg_2>, \dots) F

This function returns an iterator which runs through the list of all matrix groups <G>
in the  {\IRREDSOL} library such that
$<func_i>(G)$ lies in <arg_i>. The arguments <func_1>, <func_2>, \dots,
must be {\GAP} functions taking one argument and returning a value, and <arg_1>, <arg_2>, \dots, 
must be lists. If <arg_i> is not a list, <arg_i> is replaced by the list `[<arg_i>]'.
One of the functions must be `DegreeOfMatrixGroup' (or one of its equivalents, see below), and one
function must be  `FieldOfMatrixGroup' (or `Field'). 

For a list of functions which are handled particularly efficiently, see
`AllIrreducibleSolvableMatrixGroups' ("AllIrreducibleSolvableMatrixGroups").

Using 

`IteratorIrreducibleSolvableMatrixGroups'(<func_1>, <arg_1>, <func_2>, <arg_2>, \dots)) 

is functionally equivalent to 

`Iterator'(`AllIrreducibleSolvableMatrixGroups'(<func_1>, <arg_1>, <func_2>, <arg_2>, \dots))

(see "ref:Iterators" for details) but does not compute all relevant matrix groups at the same time. 
This may save some memory. 
For the groups returned, the attrbutes and properies described in
"Additional functionality for matrix groups" are set to their appropriate values.






%%%%%%%%%%%%%%%%%%%%%%%%%%%%%%%%%%%%%%%%%%%%%%%%%%%%%%%%%%%%%%%%%%%%%%%%%%%%%
%%
%E
%%
